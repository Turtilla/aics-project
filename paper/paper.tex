\pdfoutput=1
\documentclass[11pt]{article}
\usepackage{acl}
% Standard package includes
\usepackage{times}
\usepackage{latexsym}
\usepackage[T1]{fontenc}
\usepackage[utf8]{inputenc}
\usepackage{microtype}

\title{\textit{A young boy with a tree made of trees}: \\ Domain Adaptation of an LSTM-based Image Caption Generation Model}

\author{Dominik Künkele \\
  Master in Language Technology \\
  University of Gothenburg \\
  \texttt{guskunkdo@student.gu.se} \\\And
  Maria Irena Szawerna \\
  Master in Language Technology \\
  University of Gothenburg \\
  \texttt{gusszawma@student.gu.se} \\}

\begin{document}
\maketitle
\begin{abstract}
Here goes the abstract for the paper.
\end{abstract}

\section{Introduction}

Here we will have the introduction: some general talk, a little summary of the background reading (do we need more reading?), our questions that we want answered.

\section{Materials and Methods}

Here we need to describe the databases that we used and the code that we used, and what we added on our own. Then we need to describe how we went about evaluation.

\section{Results}

Here we should summarize our results, in terms of:
\begin{enumerate}
    \item where did the performance exceed the best prior performance (we got BEST checkpoints?)
    \item which models actually generated reasonable captions? where did they start falling apart?
    \item subjectively, which captions were best? include the questionnaire results here or subsequently
\end{enumerate}
Include figures and tables as much as we can/as much as is reasonable.

\section{Discussion}

Here we compare our results to our initial expectations and decide how they answered questions, we also contrast them with prior work (if there is any). Highlight why what we did was relevant to the field.

\section{Conclusions and further work}

Here we should summarize what we have done and suggest what more could be done on this topic.

\begin{table}
\centering
\begin{tabular}{lc}
\hline
\textbf{This is} & \textbf{a table}\\
\hline
a table & with things \\
and even & more things  \\\hline
\end{tabular}
\caption{Example table.}
\label{tab:accents}
\end{table}

% Entries for the entire Anthology, followed by custom entries
\bibliography{anthology,custom}

\appendix

\section{Example Appendix}
\label{sec:appendix}

Here we should link our repository and also make sure that it is well-organized and has proper READMEs.

\end{document}
