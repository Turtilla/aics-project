\pdfoutput=1
\documentclass[11pt]{article}
\usepackage{acl}
% Standard package includes
\usepackage{times}
\usepackage{latexsym}
\usepackage[T1]{fontenc}
\usepackage[utf8]{inputenc}
\usepackage{microtype}
\usepackage{float}

\title{\textit{A young boy with a tree made of trees}: \\ Domain Adaptation of an LSTM-based Image Caption Generation Model}

\author{Dominik Künkele \\
  Master in Language Technology \\
  University of Gothenburg \\
  \texttt{guskunkdo@student.gu.se} \\\And
  Maria Irena Szawerna \\
  Master in Language Technology \\
  University of Gothenburg \\
  \texttt{gusszawma@student.gu.se} \\}

\begin{document}
\maketitle
\begin{abstract}

Within this project we aim to explore how finetuning a CNN and LSTM-based image captioning model on a new dataset influences its performance on both the old and new data, along with what role some of the parameters, such as finetuning data size or the prevalence of out-of-vocabulary tokens in captions play in the process. We discover that differences in vocabulary play a major role and that providing too much fine-tuning data can cause the whole system to fail. We conclude that domain adaptation can be difficult to carry out, but perhaps worth attempting in certain cases.\footnote{MS} 

\end{abstract}

\section{Introduction}

\textit{Here we will have the introduction: some general talk, a little summary of the background reading (do we need more reading?), our questions that we want answered.}

Automated image caption generation is an intersectional topic, combining advances in the fields of computer vision and natural language processing. It is not only an interesting and challenging task, but also one with useful real-life implementations, such as assisting people with vision impairments in a variety of ways. It is therefore quite relevant to explore a variety of issues that can be encountered in the field, in order to improve the performance of potential models and increase their relevance in practical applications.\footnote{MS}

One potential issue with image captioning systems is that of domain adaptation, meaning adapting a trained system to somewhat different data. While it is not uncommon to encounter e.g. pre-trained image processing components in image caption generation models, it would appear that the general consensus is to train a new captioning model for every new dataset. This, however, may lead to an unnecessary waste of resources, especially if large amounts of data are used for training; another potential issue is there not always being enough data to train a model from scratch. An alternative in that case would be domain adaptation.\footnote{MS}

Some more lightweight, but well-performing image captioning architectures include ones with a CNN-based image processing component and LSTM-based language generation component; their performance can be further increased by incorporating some form of attention, which allows for a more precise identification of what is relevant to the caption generation at what point. The particular model used in this project is based on \cite{xu2015attend}, with the implementation by \cite{ilinykh}. In the model, the encoder is a convolutional network, from which representations of elements of the picture - feature vectors - are obtained. The decoder is an LSTM network with attention, determining what parts of the image are the most relevant for generating a given element of the caption (ADD WHICH KIND OF ATTENTION?).\footnote{MS} 

Another way of employing attention in a similar model is presented in \cite{lu16}, where a special "visual sentinel" helps determine when to attend to the data from the image and when to the information from the language model, as not all the words are equally represented in an image - for instance, articles or other determiners, particles, etc., stem more from the requirements of a given grammatical structure or content word rather than the scene that is being described. However, since this project did not delve into the contribution of each of these elements in caption generation, it was found to be superfluous in our particular case, even though it can lead to improved performance.\footnote{MS}

The topic of domain adaptation in caption generation itself does not appear to be widely studied. However, \cite{hesselSW15} touche upon it in their paper on the importance of the language model versus the image processing component, where they prove that a good language model can produce decent captions even with poor image input. However, what the authors actually discuss is only fine-tuning the image recognition element of the model. While that is naturally helpful, especially since models of that kind require massive amounts of training data, it does not fully address the issue of fine-tuning a whole image caption generation model, and, in fact, utilizing pre-trained CNN components has become rather common.\footnote{MS}

While all of the aforementioned papers seem to approach this issue as "generation from visual input," \cite{bernardi16} describe other approaches to the issue of image caption generation, showing that this is simply one of the possible routes. Some shortcomings of this approach that they name include the inability of the system to generate captions for visual input where the system cannot recognize any salient elements. They also tackle the issue of evaluation, listing the variety of measures that can be used to that end when it comes to caption generation (as, naturally, simpler ones like accuracy would not be a good match); they admit that human judgement or measures that mimic human judgement are key when it comes to evaluating automatically generated captions.\footnote{MS}

Given how little information we could fine on the fine-tuning of CNN+LSTM image captioning models, within this project we would like to explore the extent to which a trained model of such architecture can be adapted to another domain and how some of the hyperparameters of that process can influence the outcome. We expect this process to be effective to an extent, but it may also break the model at some point. We hope for the fine-tuned model to perform better on the images from the new domain than the non-fine-tuned one, which we aim to evaluate using human judgement. Throughout the project, Dominik focused on the influence of hyperparameters, while Maria on the evaluation, with both authors inventing ways to adapt the model to their needs and identify and fix potential issues.\footnote{MS} 

In Section 2 we describe the materials and methods used to address the aforementioned questions, such as the architecture of the model, the datasets, and the evaluation methods. In Section 3 we present the results of our inquiry. In Section 4 we discuss how they answer our questions and how they relate to previous work. Finally, in Section 5 we offer conclusions and suggestions for future projects on this topic.\footnote{MS}

\section{Materials and Methods}

\textit{Here we need to describe the databases that we used and the code that we used, and what we added on our own. Then we need to describe how we went about evaluation.}

Within this section we will describe the resources that we used in the project. We will specify the characteristics of the image captioning model, the two datasets that were used to train or fine-tune it, and the questionnaire we utilized to elicit human judgements of the best captions.\footnote{MS}

\subsection{The image captioning model \textbf{DK}}
\subsection{The code \textbf{MS and DK}}

\textit{Write about the division into files, a bit what's what, what we added and what was there. I did the evaluation part. I doubt we need to describe it in much detail, can always say that the readme has instructions and all the elements are documented, but I'd still have a bit of a summary of what we did here.}

The testing and evaluation part was conducted in a variety of Jupyter Notebook files. While the initial plan was to contain all the information in one file; unfortunately, due to GitHub limitations, that Notebook would have been too large to upload due to the number of generated ans printed image and caption pairs. Therefore, all training data sized but 2k and 5k were tested in separate notebooks using the same methods, adapted from \cite{ilinykh}, stored in a separate file, and imported into the notebook. The captions were generated using beam search and then presented together with the mappings of the attention on the image. \footnote{MS}

Other than relatively minor fixes that were required to make the code run, we decided to manually exclude the UNK token from the generation process by enforcing the lowest possible probability on it in the beam search. While this meant that we could not evaluate to what extent the models favored generating that token, it did allow us to obtain captions that sounded more natural, which was paramount for the subsequent human evaluation.\footnote{MS} 

\subsection{Flickr8k \textbf{DK}}
\subsection{imageCLEF \textbf{MS}}

The second dataset used in the project was the IAPR TC-12 Benchmark, called imageCLEF in this paper. This dataset consists of 20,000 images from a variety of real life situations. Each image is paired with a description consisting of a number of consecutive utterances specifying what the image is depicting \cite{imageclef}. The assumption made in this project is that the first fragment of the description contains the most salient elements. Unlike Flickr8k, imageCLEF contains many more images that do not include people or animals. Additionally, the way the descriptions are formulated is also different from Flickr8k.

\subsection{Human judgement questionnaire \textbf{MS}}

Having produced results (captions) using the model and the data described in the previous sections, we decided to test which of the plausible-sounding captions were the best match for the images. The only models that were somewhat consistent producing coherent captions were the non-fine-tuned (original) model and the "100" training data size models (with 4 variations of UNK filter: 0.1, 0.15, 0.2, no filter). In order to collect judgements, 20 images from 3 test sets each were collected, together with the captions generated by the aforementioned models. The test sets were: unfiltered imageCLEF, filtered (max 10\% UNK tokens) imageCLEF, and Flickr8k. The questionnaire was constructed and hosted using Google Forms.

The questionnaire itself consisted of five sections. In the first one, a modicum of personal information, such as age, gender, education, native language was collected. The following three sections were constructed the same way, and each of them corresponded to a different test set. In each of those sections, 20 images, along with the corresponding captions were presented. The order of the captions for each image was randomized to avoid bias. The participants were instructed to select one caption for each image which they thought was the best fit or the best description of the image. At the end of every section the participants had an opportunity to voice their general thoughts on the captions. Finally, in the last section the participants had the chance to include their final thoughts and opinions before submitting their answers. What is important to mention is that sometimes the model could not generate a caption for a given image, or the captions were the same across multiple models. Luckily, there were always at least two different captions. Nevertheless, this does make the interpretation of the results somewhat more difficult.

The form was subsequently distributed online, and the responses were later processed and a summary will be presented in the Results section.

\section{Results \textbf{MS and DK?}}

\textit{Here we should summarize our results, in terms of:
\begin{enumerate}
    \item where did the performance exceed the best prior performance (we got BEST checkpoints?) \textbf{DK}
    \item which models actually generated reasonable captions? where did they start falling apart? \textbf{MS or DK}
\end{enumerate}
Include figures and tables as much as we can/as much as is reasonable.}

\subsection{Hyperparameters \textbf{DK}}

\subsection{Caption Evaluation \textbf{MS}}

As described previously in the Materials and Methods section, the only models that consistently produced captions for almost all test images were the original and the "100" models. The "200" models still generated most of the captions, but many of them started being rather nonsensical or non-grammatical. This process continued in the "500" and "1000" models with fewer and fewer captions being generated successfully, and those generated ones being absolutely nonsensical. Finally in the "2k" and "5k" models most captions could not be generated.\footnote{MS} \textit{This part can be moved somewhere else if it fits better for your hyperparameters section.}

The questionnaire was filled out by 6 people. While we would like a larger sample size, we believe that we can still draw some preliminary conclusions from their answers, and that it is sufficient taking the scope of the project into account. It is also likely that the length of the questionnaire could have put off other potential respondents. The people who filled the questionnaire out ranged between 25 and 31 years of age. Three of them identified as female, two as male, one selected "other" as a response in that option. Half of the respondents had an MA/MSc degree or equivalent, two a BA/BSc or equivalent, and one selected secondary school education as the highest completed level of education. None of the people who filled the questionnaire out were native speakers of English. More detailed data about their native language was not collected. 

\begin{table}[h]
\begin{center}
\begin{tabular}{|p{1cm}|p{4.5cm}|}
\hline \bf Image & \bf Best model (\% responses) \\ \hline
15551 & 100\_0.1 (50\%)  \\
20144 & original (50\%)  \\
20343 & 100\_0.1 (33.3\%) or 100\_1.0 (33.3\%)  \\
12761 & original (33.3\%) or 100\_1.0 (33.3\%)  \\
22381 & 100\_0.1 (83.3\%) \\
10821 & 100\_0.1 (100\%) \\
18448 & original (100\%) \\
17173 & original (66.7\%) \\
13152 & original (66.7\%)  \\
20272 & original (100\%)  \\
11310 & 100\_0.1 (66.7\%)  \\
14183 & 100\_1.0 (50\%)  \\
10622 & original (50\%)  \\
13123 & original (100\%)   \\
19181 & 100\_1.0 (83.3\%) \\
16920 & 100\_0.1 (100\%) \\
11308 & original (50\%) \\
14016 & original (83.3\%) \\
21029 & original (100\%) \\
18355 & original (33.3\%) or 100\_0.15 (33.3\%) or 100\_0.2 (33.3\%) \\
\hline
\end{tabular}
\end{center}
\caption{\label{unfiltered-table} Best captions per image in the unfiltered imageCLEF test set. }
\end{table}

While the captions themselves can be seen in the Jupyter Notebook files, the aim of the questionnaire was to reveal which model or models performed the best on which test set. Therefore, the results presented here are divided between the sets and include only the information on which model's caption was deemed the best, as well as how many people voted for it. Table 1 contains the information for the unfiltered imageCLEF set, Table 2 - for the filtered one, and Table 3 for Flickr8k. Image names are the file names without the .jpg extension. In some cases more than one model's caption received the same amount of votes, or the winning caption was generated by more than one model. 

\begin{table}[h]
\begin{center}
\begin{tabular}{|p{1cm}|p{4.5cm}|}
\hline \bf Image & \bf Best model (\% responses) \\ \hline
40416 & original (83.3\%)  \\
39158 & original (50\%)  \\
25053 & original (83.3\%)  \\
30620 & 100\_1.0 (66.7\%)  \\
32397 & 100\_0.2 (66.7\%) \\
38937 & original (50\%) \\
39005 & 100\_0.1 (33.3\%) or 100\_0.15 (33.3\%) or 100\_0.2 (33.3\%) \\
40120 & original (66.7\%) \\
23588 & 100\_1.0 (83.3\%)  \\
39472 & 100\_0.1 (66.7\%)  \\
30138 & original (50\%)  \\
40202 & 100\_0.1 (83.3\%)  \\
35895 & original (66.7\%)  \\
32663 & 100\_0.2 (83.3\%)   \\
30705 & original (83.3\%) \\
38081 & 100\_0.15 (66.7\%) \\
31571 & original (100\%) \\
35858 & original (66.7\%) \\
37836 & 100\_0.1 (33.3\%) or 100\_0.15 (33.3\%) \\
39239 & 100\_0.15, 100\_0.2, and 100\_1.0 (50\%)\footnote{These models generated identical captions.} \\
\hline
\end{tabular}
\end{center}
\caption{\label{filtered-table} Best captions per image in the filtered imageCLEF test set. }
\end{table}

As can be noticed in Table 1, the majority of the "good" captions for the unfiltered imageCLEF set were generated by the original, non-fine-tuned model, or by the one with a 10\% filter. Some captions from the model fine-tuned on nonfiltered data turned out to have been rated high as well. Finally, other models' captions were rated as good when they overlapped with the original model's caption. Overall, while the judgements are split rather evenly, the original model seems to generate better captions for this test set. It is worth pointing out that relatively few captions were unanimously voted best. 

When it comes to the filtered imageCLEF test set (Table 2), where a maximum of 10\% tokens in the original caption could be unknown tokens, the spectrum of what models' captions were considered good is even wider. There was only one case where one caption was voted best by all the participants. Half of the best captions were generated by the original model, and half by various fine-tuned ones, with none of them being a clear front-runner in this case. 

Finally, when it comes to the original Flickr8k-based test set, as seen in Table 3, the original model without fine-tuning performed best, although there were instances where the 10\% or 20\% fine-tuned models generated better captions. Many of the original model's captions were also chosen unanimously. 

\begin{table}[h]
\begin{center}
\begin{tabular}{|p{2.5cm}|p{3cm}|}
\hline \bf Image & \bf Best model (\% responses) \\ \hline
2116444946[...] & original (100\%)  \\
2316097768[...] & original (100\%)  \\
2439384468[...] & original (66.7\%)  \\
2112921744[...] & 100\_0.2 (50\%)  \\
2392460773[...] & original (100\%) \\
2434006663[...] & original (100\%) \\
2308256827[...] & original (100\%) \\
2111360187[...] & original (66.7\%) \\
2271671533[...] & original (66.7\%)  \\
2328616978[...] & original (100\%)  \\
2456907314[...] & original (66.7\%)  \\
2229179070[...] & 100\_0.1 (100\%)  \\
2279980395[...] & 100\_0.1 (88.3\%)  \\
2393971707[...] & original (83.3\%)   \\
211277478[...] & 100\_0.1 (50\%) \\
2337919839[...] & original (83.3\%) \\
2447035752[...] & original (50\%) \\
23445819[...] & 100\_0.2 (66.7\%) \\
2448210587[...] & original (33.3\%) \\
2445654384[...] & 100\_0.1 (50\%) or 100\_0.2 (50\%)\\
\hline
\end{tabular}
\end{center}
\caption{\label{flickr-table} Best captions per image in the Flickr8k test set. }
\end{table}

When it comes to the comments left by the respondents after each section, they were not very positive. The captions in the first content section were judged as "unfitting," "strange," "ungrammatical," "inaccurate," "nonsensical," or "confusing," with some of the respondents admitting that the mismatch between the image and captions made them anxious or made their "brain hurt." For the second content section, some people considered the captions better, and some worse than previously, and the feeling of anxiety persisted here too. In addition, half of the respondents pointed out that the captions tend to include people in the descriptions even when there are no people in the images. It seems that the impressions of the captions in the last content section were better: while people said that they were still "strange," "weird," or "confusing," they also admitted that the verbs in the sentences tended to be accurate descriptions of the events in the images, and that they were overall "somewhat accurate." The person who expressed their feelings of anxiety previously now said that they "started to get used to them," which could also be a consequence of there being a smaller mis-match between the images and the captions. 

In the final remarks many of the respondents expressed their dissatisfaction with the overall quality of the captions, leaving comments such as "r u b b i s h" or "my brain melted." Two people mentioned again how the captions would include people despite there not being any in the images. One person said that the simpler captions or images were clear, while the more complex ones were problematic or mismatched. Another person said that these "didn't sound like human-written captions," and yet another that they oftentimes "missed the essence of the picure."  

Overall, the original model performed best on the original test set. The most captions generated by fine-tuned models were selected as best for the filtered imageCLEF test set. The unfiltered test set was also fairly balanced between the fine-tuned and original models. While for the unfiltered data the unfiltered and 10\% fine-tuned models can be identified as the best out of the fine-tuned ones, the same is impossible to determine for the filtered test set. From these results it can be concluded that while fine-tuning does improve the quality of the captions somewhat (both for imageCLEF and Flickr8k), it does not have that effect for every image. Judging by the open responses from the participants, the captions generated for Flickr8k were the least wrong, but their quality was still low, and the ones for imageCLEF were even worse. 

Clearly, while some degree of improvement can be obtained by this kind of fine-tuning, it is far from perfect and the captions rarely match the images. This is likely due to the disparity between the kinds of images in the two sets. Another clue to that being a major issue is the models' tendency to - as pointed out by the respondents - insert people into captions when there were none in the original image. A large portion of the captions in the Flickr8k dataset are constructed that way, and it must be one of the things that the model learned to try to identify in the image at all cost and feature in the caption, and that the fine-tuning did not help mitigate.

Finally, it is worth keeping in mind that some of the apparent weirdness of the captions may stem from the relatively small vocabulary that the models had, which was likely lacking the proper words to describe certain elements in the images, even when the ones whose captions contained many new words were filtered out. 

\section{Discussion \textbf{MS and/or DK}}

\textit{Here we compare our results to our initial expectations and decide how they answered questions, we also contrast them with prior work (if there is any). Highlight why what we did was relevant to the field.}

\section{Conclusions and further work \textbf{MS and/or DK}}

\textit{Here we should summarize what we have done and suggest what more could be done on this topic.}

\begin{table}
\centering
\begin{tabular}{lc}
\hline
\textbf{This is} & \textbf{a table}\\
\hline
a table & with things \\
and even & more things  \\\hline
\end{tabular}
\caption{Example table.}
\label{tab:accents}
\end{table}

% Entries for the entire Anthology, followed by custom entries
\bibliography{anthology,custom}

\appendix

\section{Example Appendix}
\label{sec:appendix}

\textit{Here we should link our repository and also make sure that it is well-organized and has proper READMEs.}

\end{document}
